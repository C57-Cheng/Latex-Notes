\section{真空中的静电场}

库仑定律:
$$\displaystyle F=\frac{1}{4\pi\varepsilon_0}\frac{q_1q_2}{r^2}
\text{,其中,}\ \varepsilon_0=8.85\times 10^{-2}\ C^2/(N\cdot m^2)$$

场强公式:
$$\displaystyle E=\frac{F}{q_0}=\frac{1}{4\pi \varepsilon_0 \cdot \dfrac{q}{r^2}}$$

电偶极矩:
$$
p_e=ql
$$

高斯定理:
$$
\oint E \di S = \frac{q}{\varepsilon_0}
$$

电势:
$$
U_p=\int_{p}^{\infty}E\cdot\di l =\int_{r}^{\infty}\frac{q}{4\pi \varepsilon_0 r^2}\di r
=\frac{q}{4\pi \varepsilon_0 r}
$$

电势与场强的关系:
$$
E=-(\frac{\partial U}{\partial x}\boldsymbol{i}+\frac{\partial U}{\partial y}\boldsymbol{j}+\frac{\partial U}{\partial z}\boldsymbol{k})
=-grad \ U
$$

\newpage

\section{静电场中的导体和电介质}

静电屏蔽

电容:
$$
C=\frac{Q}{U}
$$
$$
U=Ed=\frac{\sigma d}{\varepsilon_0 }=\frac{qd}{\varepsilon_0S}\Rightarrow C=\frac{\varepsilon_0S}{d}
$$

静电场中的电介质:
$$
E=\frac{u}{d}=\frac{U_0}{\varepsilon_r d}=\frac{E_0}{\varepsilon_r}
$$

极化强度:
$$
\boldsymbol{P}=\varepsilon_0 \chi_e \boldsymbol{E}
$$

极化电荷面密度:
$$
\sigma ' =\left| \boldsymbol{P}\right| \cos \theta=\boldsymbol{P}\cdot \boldsymbol{e_n}
$$
$$
\begin{aligned}
    \vec{E}=\vec{E_0}+\vec{E'}
    &\Rightarrow E=E_0-E'=\frac{\sigma_0 }{\varepsilon_0}-\frac{\sigma'}{\varepsilon_0}
    =E_0-E'=\frac{\sigma_0 }{\varepsilon_0}-\frac{P}{\varepsilon_0}=\frac{\sigma_0}{\varepsilon_0}-\chi_e E \\
    &\Rightarrow E =\frac{E_0}{1+\chi_e}\quad \text{定义}\quad 1+\chi_e=\varepsilon_r \Rightarrow E=\frac{E_0}{\varepsilon_r}\\
    &\Rightarrow \frac{\sigma_0-\sigma'}{\varepsilon_0}=\frac{\sigma_0}{\varepsilon_0 \varepsilon_r}=\frac{\sigma_0}{\varepsilon}\\
    &\Rightarrow \sigma'=(1-\frac{1}{\varepsilon_r})\sigma_0
\end{aligned}
$$

电介质中的高斯定理:
$$
\oint E\cdot \di S = \frac{1}{\varepsilon_0} (\sum q_0+\sum q')\\
$$
$$
\oint E\cdot \di S = \frac{1}{\varepsilon_0}(\sigma_0\Delta S - \sigma'\Delta S)\\
$$
$$
\oint E\cdot \di S = \frac{\sigma_0\Delta S}{\varepsilon_0} - \frac{1}{\varepsilon_0} \oint P \di S\\
$$
$$
\oint (\varepsilon_0E+P)\di S = \sum q_0\\
$$
$$
\Rightarrow \oint D \di S =\sum q_0
$$